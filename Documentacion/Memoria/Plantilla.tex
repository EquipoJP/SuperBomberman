\documentclass[a4paper]{article}

\usepackage[utf8]{inputenc}
\usepackage[spanish]{babel}
\usepackage{graphics}
\usepackage{caption}
\usepackage{subcaption}
\usepackage[demo]{graphicx}
\usepackage{enumitem}
\usepackage{longtable}
\usepackage{listings}
\usepackage{listingsutf8}
\usepackage{framed}
\usepackage{float}
\usepackage{hyperref}
\usepackage{amsmath}

\begin{document}

\begin{titlepage}

\begin{center}
\vspace*{1.5in}
\vspace*{-1in}
\begin{figure}[htb]
\begin{center}
\includegraphics[width=8cm]{logoUZ.png}
\end{center}
\end{figure}

\vspace*{0.3in}

Universidad de Zaragoza \\

\vspace*{0.3in}

\begin{large}
VIDEOJUEGOS\\
\end{large}
\vspace*{0.2in}
\begin{Large}
\textbf{Clon de Bomberman} \\
\end{Large}
\vspace*{0.3in}
\begin{large}
\end{large}
\vspace*{0.1in}
\rule{80mm}{0.1mm}\\
\vspace*{0.1in}
\begin{large}
Hecho por: \\
Jaime Ruiz-Borau Vizárraga \\
Patricia Lázaro Tello \\

\end{large}
\end{center}

\end{titlepage}
\tableofcontents

\newpage
\section{Descripción del videojuego realizado}
\paragraph{}Como se describe en el título de la memoria, el videojuego clásico elegido para la elaboración de un clon es el \textbf{Bomberman}. Dada la elevada similitud de mecánica, enemigos y objetivos entre las distintas entregas de Bomberman, se optó por implementar la mecánica del \textbf{primer Bomberman} que salió al mercado. Sin embargo, los sprites y las imágenes empleadas para el juego son de entregas de Bomberman posteriores.
\vspace*{0.2in}
\begin{figure}[H]
	\centering
	\begin{minipage}[b]{0.4\textwidth}
		\includegraphics[width=\textwidth]{primerBomberman.png}
		\caption{Bomberman clásico}
	\end{minipage}
	\hfill
	\begin{minipage}[b]{0.4\textwidth}
		\includegraphics[width=\textwidth]{bomberman.png}
		\caption{Clon desarrollado}
	\end{minipage}
	\label{fig:primerBomberman}
\end{figure}
La única diferencia con el original es que la puntuación no se muestra hasta el final de la partida. También se ha implementado un menú con opciones para regular el volumen de la música del juego y la posibilidad de configurar los controles del juego.
\section{Detalles de implementación en 2D}
\paragraph{}Para el juego en 2D se implementó un motor gráfico basado en un hilo principal renderizando frames a razón de 60 por segundo mostrando los distintos sprites del juego utilizando las funciones estándar de Java para mostrar gráficos por pantalla.
\newpage
\begin{thebibliography}{99} 
\bibitem{paraQueSirve} \textbf{Title} - Consulted in XXXXX aaaa. [\url{link}]
\bibitem{fig:primerBomberman} \textbf{Imágenes del primer Bomberman} - By Source, Fair use, [\url{https://en.wikipedia.org/w/index.php?curid=12465479}]

\end{thebibliography}

\end{document}